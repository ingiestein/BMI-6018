
%!TEX TS-program = xelatex
%!TEX encoding = UTF-8 Unicode

\documentclass[12pt]{extarticle}

\usepackage{geometry}                % See geometry.pdf to learn the layout options. There are lots.
\geometry{letterpaper}                   % ... or a4paper or a5paper or ... 
%\usepackage[parfill]{parskip}    % Activate to begin paragraphs with an empty line rather than an indent
\usepackage{graphicx}
\usepackage{amssymb}
\usepackage{hyperref}
\usepackage{setspace}
\usepackage[]{fancyhdr}
\doublespacing
% Will Robertson's fontspec.sty can be used to simplify font choices.
% To experiment, open /Applications/Font Book to examine the fonts provided on Mac OS X,
% and change "Hoefler Text" to any of these choices.

\usepackage{fontspec,xltxtra,xunicode}

\defaultfontfeatures{Mapping=tex-text}
\setromanfont[Mapping=tex-text]{Hoefler Text}
\setsansfont[Scale=MatchLowercase,Mapping=tex-text]{Gill Sans}
\setmonofont[Scale=MatchLowercase]{0xProto Nerd Font Mono}

\setcounter{secnumdepth}{0} % Numbers up to \subsection



\pagestyle{fancy}
\setlength{\headheight}{17.0pt}
\fancyfoot[L]{Stein Ingebretsen M.D.}
\fancyhead[C]{Biomedical Informatics Personal Statement}
\fancyfoot[R]{\today}
\title{Biomedical Informatics Personal Statement}
\author{Stein Ingebretsen}
\date{\today}                                           % Activate to display a given date or no date
\ttfamily

\begin{document}

In the course of human history there have been a number of technological advancements that shook the foundations of the world: Bronze, steel, the printing press, the cotton gin, the atomic bomb, the computer and the internet. For our generation it will be generative AI and LLMs. In the few years it has been accessible to the public it has already altered both private life and the corporate world.  

Medicine is no different. Medicine has come a long way from blood letting and the releasing of~ ‘humors.’ We live in an era of clinical research and medical students are taught from their first day of school to practice evidence based medicine. Fields such as oncology and cardiology are among the most rigourously studied, protocolized, and algorhythmic. Others, such as rheumatology or radiology have a greater degree of freedom and subjective interpretation as they perform their duties. With the advent of AI and LLMs, all physicians in all fields of medicine find themselves at a tipping point; either the physicians adapt and advance with the world, or fall behind. 

The medical world needs to embrace and be pioneers in this new technology. In my practice as a hospitalist, AI is already being used to summarize doctor-patient interactions by summarizing the recorded audio into a pertinent medical history. This in turn offloads some of the clerical work from our physicians, allowing them to focus on patient care, rather than documentation. There are phone applications which have trained LLMs on research data with the intent to assist clinicians understand the current research and best practices. When asked AI can even generate a differential diagnosis for a symptom, though they lack in discernment. As a professional in the field, it is easy to see where LLMs fall very short of the mark. They can only regurgitate the data on which they have been trained. They can tell you guidlines, best practices, and diagnostic work ups as recommended by the major medical associations around the country, but they lack empathy, interpersonality, creativity, and nuanced clinical analysis of a given patient.

Despite their drawbacks, LLMs have growing potential, particuarly in data analytics. It's imperative for physicians and public health officials to understand disease trends in different regions of our country. For example processing and analyzing data and trends in melanoma, colon cancer, heart disease, alcoholism, substance abuse, viral infections and hundreds of other chronic and acute medical conditions in our population is where AI and LLMs can start improving patient care. It is difficult for humans to assimilate large amounts of data quickly, and see interrelationships between the data. This is where AI can excell. It can process medical data along numerous different metrics potentially providing insight into the genomic, epidimiologic, or physiologic research opportunities for countless disease processes. It can make primary care providers more efficient and more effective by providing treatment plan recommendations or disease work-up pathways. It can identify a rash in the clinic, lung cancer earlier on chest imaging, and help physicians anticipate trends in viral spread.

We are at a tipping point in medicine, and physicians who don't embrace AI will be left behind. I want to be at the tip of the spear in the transition from how medicine was practiced, to how it will be practiced. Gaining undestanding in AI models in bioinformatics and data science is an obvious choice to achieve my goal. 
\end{document}

